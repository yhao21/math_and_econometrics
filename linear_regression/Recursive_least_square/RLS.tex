
\documentclass[10pt]{article}
\usepackage{amsmath}
\DeclareMathOperator*{\argmin}{arg\,min} % thin space, limits underneath in displays
\DeclareMathOperator*{\argmax}{arg\,max} % thin space, limits underneath in displays
\newtheorem{thm}{Theorem}
\usepackage{amssymb}
\usepackage{amsfonts}
\usepackage{mathrsfs}
\usepackage{bm}
\usepackage{indentfirst}
\setlength{\parindent}{0em}
\usepackage[margin=1in]{geometry}
\usepackage{graphicx}
\usepackage{setspace}
\doublespacing
\usepackage[flushleft]{threeparttable}
\usepackage{booktabs,caption}
\usepackage{float}
\usepackage[sort,comma]{natbib}
\usepackage[hidelinks]{hyperref}
\usepackage{booktabs}
\usepackage{multirow}

\usepackage{import}
\usepackage{xifthen}
\usepackage{pdfpages}
\usepackage{transparent}

\newcommand{\incfig}[1]{%
\def\svgwidth{\columnwidth}
\import{./figures/}{#1.pdf_tex}
}




\title{}
\author{}
\date{}


\begin{document}
\maketitle


Section 3.2 gives an introduction about detail models and model parametrization.

\section{Intro}

There are basically two approaches to the problem of building a mathematical model
of a given system.
One, based on the physical laws and relationships that (are supposed to) govern the
system's behavior, a mathematical model can be constructed. This procedure is
called {\textbf {modeling}}. Modeling can be quite time-consuming and may lead to models
that are unnecessarily complex.

Two, signals produced by the  system can be measured and be used to construct a model.
We call it {\textbf {identification}}.
Identification could mean that a batch of data is collected from the system, and that
subsequently, this batch of data is used to construct a model. Such a procedure is 
usually called {\textbf {off-line identification}} or {\textbf {batch identification}}
(You feed all data into the model and get an estimate).


We use the {\textbf {recursive identification}} for a procedure which is to infer the
model at the same time as the data is collected. The model is then updated at each time
instant some new data becomes available (Synonymous terms are on-line identification,
real-time identification, adaptive algorithm, and sequential estimation).




\subsection{Shortcomings of recursive identification}
\begin{itemize}
\item The decision of what model structure to use has to be made a priori.
\item Recursive methods do not give as good accuracy of the models as off-line methods.
		For long data records, the difference need not be significant.
\end{itemize}





\bibliographystyle{plainnat}
\bibliography{my_bib}

\end{document}

