
\documentclass[12pt]{article}
\usepackage{amsmath}
\DeclareMathOperator*{\argmin}{arg\,min} % thin space, limits underneath in displays
\DeclareMathOperator*{\argmax}{arg\,max} % thin space, limits underneath in displays
\newtheorem{thm}{Theorem}
\usepackage{amssymb}
\usepackage{amsfonts}
\usepackage{mathrsfs}
\usepackage{bm}
\usepackage{indentfirst}
\setlength{\parindent}{0em}
\usepackage[margin=1in]{geometry}
\usepackage{graphicx}
\usepackage{setspace}
\doublespacing
\usepackage[flushleft]{threeparttable}
\usepackage{booktabs,caption}
\usepackage{float}
\usepackage{graphicx}
\usepackage[sort,comma]{natbib}
\usepackage[hidelinks]{hyperref}

\usepackage{import}
\usepackage{xifthen}
\usepackage{pdfpages}
\usepackage{transparent}

\newcommand{\incfig}[1]{%
\def\svgwidth{\columnwidth}
\import{./figures/}{#1.pdf_tex}
}




\title{Modeling}
\author{}
\date{}


\begin{document}
\maketitle

Task of modeling: choose the most appropriate family (density) for the data.
\begin{itemize}
\item 1. Define the probability model in terms of unknown parameters $ \bm{\theta} $, and let
	the data.
\item 2. Make statistical inference, choose its appropriate value from parameter space 
	$ \bm{\Theta} $.
\end{itemize}
How do we make the original decision regarding which probability model is appropriate?
\begin{itemize}
\item 1. Compare the distribution shapes to the histogram of the observed data.
	\begin{itemize}
	\item What if two distributions have very similar shape?
	\item The best way to distinguish them is via index measures based on moments. 
	\end{itemize}
\item 2. In addition to the shapes and parameters, one should consider the {\textbf {support}}
	of the density.
	\begin{itemize}
	\item Consider Beta model, the support is from 0 to 1. It is useful to model the exam scores
		(scores are from 0\% to 100\%).
	\item It will not be a good idea to use Normal distribution to model exam scores, because the
		the support of the Normal distribution is from $  - \infty  $ to $  + \infty  $.
	\end{itemize}
\end{itemize}




\bibliographystyle{plainnat}
\bibliography{my_bib}

\end{document}

