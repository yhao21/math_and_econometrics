
\documentclass[10pt]{article}
\usepackage{amsmath}
\DeclareMathOperator*{\argmin}{arg\,min} % thin space, limits underneath in displays
\DeclareMathOperator*{\argmax}{arg\,max} % thin space, limits underneath in displays
\newtheorem{thm}{Theorem}
\usepackage{amssymb}
\usepackage{amsfonts}
\usepackage{mathrsfs}
\usepackage{bm}
\usepackage{indentfirst}
\setlength{\parindent}{0em}
\usepackage[margin=1in]{geometry}
\usepackage{graphicx}
\usepackage{setspace}
\doublespacing
\usepackage[flushleft]{threeparttable}
\usepackage{booktabs,caption}
\usepackage{float}
\usepackage[sort,comma]{natbib}
\usepackage[hidelinks]{hyperref}
\usepackage{booktabs}
\usepackage{multirow}

\usepackage{import}
\usepackage{xifthen}
\usepackage{pdfpages}
\usepackage{transparent}

\newcommand{\incfig}[1]{%
\def\svgwidth{\columnwidth}
\import{./figures/}{#1.pdf_tex}
}




\title{Matrix Calculus}
\author{}
\date{}


\begin{document}
\maketitle
\tableofcontents


\section{Matrix transpose}

\begin{itemize}
\item 
For matrix $ \bm{A} $ and $ \bm{B} $, and a scalar $ c $,
\begin{align*}
(\bm{A}')' &= \bm{A}\\
(\bm{A} + \bm{B})' &= \bm{A}' + \bm{B}'\\
(c\bm{A})' &= c\bm{A}'\\
(\bm{A}\bm{B})'&= \bm{B}'\bm{A}'
\end{align*}
\item 
Given matrices $ \bm{A}_{1}, \bm{A}_{2},..., \bm{A}_{n} $, 
\begin{equation}
(\bm{A}_{1}\bm{A}_{2}\cdots \bm{A}_{n})' = \bm{A}_{n}'\bm{A}_{n - 1}'\cdots \bm{A}_{1}'
\end{equation}

\item 
		\begin{equation}
		det(\bm{A}') = det(\bm{A})
		\end{equation}
\item 
		\begin{equation}
		(\bm{A}')^{ - 1} = (\bm{A}^{ - 1})'
		\end{equation}


\end{itemize}


\section{Derivatives of the trace of a matrix}
\subsection{Matrix and index notation}
Given the product of matrix $ \bm{A} $ and $ \bm{B} $ is
a $ i  \times k $ matrix, $ [ \bm{AB}]_{ik} $,
we can write it in the index form as below,
\begin{equation}
		[\bm{AB}]_{ik} = \sum\limits_{j}A_{ij}B_{jk}.
\end{equation}
And the matrix product $ \bm{ABC}' $ can be written as
\begin{equation}
[\bm{ABC}']_{il} = \sum\limits_{j}A_{ij}[\bm{BC}']_{jl}
=\sum\limits_{j}A_{ij} \sum\limits_{k}B_{jk}C_{kl}'
=\sum\limits_{j}A_{ij} \sum\limits_{k}B_{jk}C_{lk}
=\sum\limits_{j} \sum\limits_{k}A_{ij}B_{jk}C_{lk}
\end{equation}


\subsection{First-order derivatives of the trace}
The trace of a matrix, $ tr(\cdot ) $, is the summation
of elements on the main diagonal of a square matrix,
i.e., it must be a $ n  \times  n $ matrix.


{\textbf {Example 1}}

Consider this example:
\begin{equation}
		f = tr[\bm{AXB}],
\end{equation}
we can write this using index notation as below,
\begin{align*}
		f &= \sum\limits_{i} [\bm{AXB}]_{ii} \\
			&= \sum\limits_{i} \sum\limits_{j} A_{ij}[\bm{XB}]
			_{ji}\\
			&= \sum\limits_{i} \sum\limits_{j} A_{ij}
			\sum\limits_{k} X_{jk}B_{ki}\\
			&= \sum\limits_{i} \sum\limits_{j} \sum\limits_{k}
			A_{ij}X_{jk}B_{ki}.
\end{align*}


Taking the derivative w.r.t. $ X_{jk} $, we get
\begin{align}
		\frac{\partial tr[AXB] }{\partial X_{jk} } &= 
\sum\limits_{i} A_{ij}B_{ki} \label{eqn: Kjk}\\
&= \sum\limits_{i} B_{ki}A_{ij} \label{eqn:BA}\\
&= [\bm{BA}]_{kj}, \text{ it is a $ k  \times j $ matrix }
\end{align}

The summation of $ j $ and $ k $ disappears in equation \eqref{eqn: Kjk} becase
we are differenciating the trace w.r.t. $ X_{jk} $. We switch the position in 
equation \eqref{eqn:BA} following $ k  \times  i $ times $ i  \times j $ rule.





The result has to be the same size as $ \bm{X} $, 
i.e., a $ j  \times  k $ matrix (We differentiate the
trace w.r.t. $ X_{jk} $ in equation \eqref{eqn: Kjk}). Hence we have to transpose the result. Theresore,
\begin{equation}
		\frac{\partial tr[\bm{AXB}] }{\partial \bm{X} }
		=\bm{A}'\bm{B}' \label{eqn:axb}
\end{equation}



{\textbf {Example 2}}
Given  matrix $ [\bm{AX}'\bm{B}]_{ii} $, the trace of it is
\begin{align*}
		tr[\bm{AX}'\bm{B}] &= \sum\limits_{i}\sum\limits_{j}
		A_{ij}[\bm{X'}\bm{B}]_{ji}\\
		&= \sum\limits_{i}\sum\limits_{j}A_{ij}
		\sum\limits_{k} X_{jk}'B_{ki}\\
		&= \sum\limits_{i}\sum\limits_{j}A_{ij}
		\sum\limits_{k} X_{kj}B_{ki}\\
		&= \sum\limits_{i}\sum\limits_{j}\sum\limits_{k}A_{ij}
		 X_{kj}B_{ki}.
\end{align*}

Differentiate $ tr[\bm{AX}'\bm{B}] $ w.r.t. $ X_{kj} $,
\begin{align*}
\frac{\partial tr[\bm{AX}'\bm{B}] }{\partial X_{kj} }
&= \sum\limits_{i} A_{ij}B_{ki}\\
&= \sum\limits_{i} B_{ki}A_{ij}\\
&= [\bm{BA}]kj
\end{align*}
Since $ [\bm{BA}]_{kj} $ and $ \bm{X} $ already have the same size, i.e., $ k  \times j $, we do not need to transpose the matrix. Therefore,
\begin{equation}
		\frac{\partial tr[\bm{AX}'\bm{B}] }{\partial \bm{X} } = \bm{BA} \label{eqn:AX'B}
\end{equation}



\subsection{Multiple-order}
Now consider the trace of a matrix with two $ \bm{X} $,
\begin{align}
		f &= tr[\bm{AXBXC}']\\
		&= \sum\limits_{i}\sum\limits_{j}\sum\limits_{k}\sum\limits_{l}\sum\limits_{m}
		A_{ij}X_{jk}B_{kl}X_{lm}C_{im}.
\end{align}
Since we have two terms related to $ \bm{X} $, we need to differentiate the trace
w.r.t. both of them, i.e., 
\begin{equation*}
\frac{\partial f }{\partial \bm{X} } = \frac{\partial f }{\partial X_{jk} }
 + \frac{\partial f }{\partial X_{lm} }.
\end{equation*}


Differentiate $ f $ w.r.t. $ X_{jk} $,
\begin{align}
\frac{\partial f }{\partial X_{jk} }&= \sum\limits_{i}\sum\limits_{l}\sum\limits_{m}
A_{ij}B_{kl}X_{lm}C_{im}\\
&= \sum\limits_{i}\sum\limits_{l}\sum\limits_{m}A_{ij}B_{kl}X_{lm}C_{mi}'		\\
&= \sum\limits_{i}\sum\limits_{l}A_{ij}B_{kl}[\bm{XC}']_{li}		\\
&= \sum\limits_{i}A_{ij}[\bm{BXC}']_{ki}		\\
&= \sum\limits_{i}[\bm{BXC}']_{ki}A_{ij}		\\
&= [\bm{BXC}'\bm{A}]_{kj}\\
\frac{\partial f }{\partial X_{jk} }&= \bm{A}'\bm{C}\bm{X}'\bm{B}'
\end{align}

Differentiate $ f $ w.r.t. $ X_{lm} $,
\begin{align}
\frac{\partial f }{\partial X_{lm} }
&=\sum\limits_{i}\sum\limits_{j}\sum\limits_{k}A_{ij}X_{jk}B_{kl}C_{im}\\
&=\sum\limits_{i}\sum\limits_{j}A_{ij}[\bm{XB}]_{jl}C_{im}\\
&=\sum\limits_{i}[\bm{AXB}]_{il}C_{im}\\
&=\sum\limits_{i}[\bm{AXB}]_{il}C_{mi}'\\
&=\sum\limits_{i}C_{mi}'[\bm{AXB}]_{il}\\
&=[\bm{C}'\bm{AXB}]_{ml}\\
\frac{\partial f }{\partial X_{lm} }&= \bm{B}'\bm{X}'\bm{A}'\bm{C}
\end{align}


Hence,
\begin{equation}
		\frac{\partial tr[\bm{AXBXC}'] }{\partial \bm{X} } = \bm{A}'\bm{C}\bm{X}'\bm{B}'
		 + \bm{B}'\bm{X}'\bm{A}'\bm{C}.
\end{equation}


{\textbf {Notice}}, we can do it in an easier way.
First, we decomposite $ \bm{AXBXC}' $ into two parts regarding the position of two 
$ \bm{X} $, i,e., $ \bm{AXD} $ and $ \bm{EXC}' $, where $ \bm{D} =\bm{BXC}' $,
$ \bm{E} = \bm{AXB} $.
Thus,
\begin{equation*}
		\frac{\partial tr[\bm{AXBXC}'] }{\partial \bm{X} } = 
		\frac{\partial tr[\bm{AXD}] }{\partial \bm{X} } +
		\frac{\partial tr[\bm{EXC}'] }{\partial \bm{X} }.
\end{equation*}

Apply the result we got from equation \eqref{eqn:axb}, i.e., 
$ \frac{\partial tr[\bm{AXB}] }{\partial \bm{X} }=\bm{A}'\bm{B}' $

\begin{align*}
		\frac{\partial tr[\bm{AXBXC}'] }{\partial \bm{X} } &=
		\frac{\partial tr[\bm{AXD}] }{\partial \bm{X} } +
		\frac{\partial tr[\bm{EXC}'] }{\partial \bm{X} }\\
		&= \bm{A}'\bm{D}' + \bm{E}'\bm{C}\\
		&= \bm{A}'\bm{C}\bm{X}'\bm{B}' + \bm{B}'\bm{X}'\bm{A}'\bm{C}
\end{align*}

where 
\begin{align*}
\bm{D} &= \bm{BXC}'\\
\bm{E} &= \bm{AXB}
\end{align*}


{\textbf {One more example:}}

Now consider this even more complicated one,
\begin{equation*}
		f = tr[\bm{AX}\bm{X}'\bm{BCX}'\bm{XC}].
\end{equation*}
First, we decomposite it into four parts because there are four terms related to
$ \bm{X} $s,
\begin{align*}
&\bm{AXD}\\
&\bm{E}\bm{X'}\bm{F}\\
&\bm{GX}'\bm{H}\\
&\bm{IXC},
\end{align*}
where
\begin{align*}
\bm{D} &=  \bm{X}'\bm{BCX}'\bm{XC}  \\
\bm{E} &=  \bm{AX} \\
\bm{F} &=  \bm{BCX}'\bm{XC}  \\
\bm{G} &=  \bm{AX}\bm{X}'\bm{BC}  \\
\bm{H} &=  \bm{XC}  \\
\bm{I} &=  \bm{AX}\bm{X}'\bm{BCX}'  \\
\end{align*}

Then we write down the derivatives
\begin{align*}
\frac{\partial tr[\bm{AX}\bm{X}'\bm{BCX}'\bm{XC}] }{\partial \bm{X} }
&= \frac{\partial tr[\bm{AXD}] }{\partial \bm{X} }
+\frac{\partial \bm{E}\bm{X'}\bm{F} }{\partial \bm{X} }
+\frac{\partial \bm{GX}'\bm{H} }{\partial \bm{X} }
+\frac{\partial \bm{IXC} }{\partial \bm{X} }.
\end{align*}

Apply the results from equation \eqref{eqn:axb} and equation \eqref{eqn:AX'B},
i.e., $ \frac{\partial tr[\bm{AXB}] }{\partial \bm{X} }=\bm{A}'\bm{B}' $ and
$ \frac{\partial tr[\bm{AX}'\bm{B}] }{\partial \bm{X} } = \bm{BA} \label{eqn:AX'B} $

\begin{align*}
\frac{\partial tr[\bm{AX}\bm{X}'\bm{BCX}'\bm{XC}] }{\partial \bm{X} }
&= \frac{\partial tr[\bm{AXD}] }{\partial \bm{X} }
+\frac{\partial \bm{E}\bm{X'}\bm{F} }{\partial \bm{X} }
+\frac{\partial \bm{GX}'\bm{H} }{\partial \bm{X} }
+\frac{\partial \bm{IXC} }{\partial \bm{X} }\\
&= \bm{A}'\bm{D}' + \bm{FE} + \bm{HG} + \bm{I}'\bm{C}'\\
&= \bm{A}'\bm{C}'\bm{X}'\bm{X}\bm{C}'\bm{B}'\bm{X} + \bm{BCX}'\bm{XCAX}
 + \bm{XCAX}\bm{X}'\bm{BC} + \bm{X}\bm{C}'\bm{B}'\bm{X}\bm{X}'\bm{A}'\bm{C}'
\end{align*}



		 
















\bibliographystyle{plainnat}
\bibliography{my_bib}

\end{document}

